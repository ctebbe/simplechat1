%
% Caleb Tebbe & Zach Kaplan CS314 P2T.tex
% copy-paste this code in www.writelatex.com to see a compiled version
%
\documentclass[12pt]{article}

\usepackage[english]{babel}
\usepackage[utf8x]{inputenc}


\title{CS314 Simplechat P2 Testcases}
\author{Caleb Tebbe \& Zach Kaplan}
\date{April 1, 2013}


\begin{document}
\maketitle


% new test case, section automatically increments numbers
% test case 1
\section{Testcase D1001}
\begin{itemize}
\item System: SimpleChat
\item Phase: 3
\item Description: Test private messages between clients.
\end{itemize}

% * disables numbering
\subsection*{Rationale}
Test to see if a client has the ability to send a private message to another connected client.

\subsection*{Instructions}
\begin{enumerate}
\item Start server
\item Start a client logged in as test
\item Start a client logged in as test2
\item Start a client logged in as test3
\item Use command ``\#private test2 test message'' from client test
\end{enumerate}

\subsection*{Expected Results}
\begin{enumerate}
\item Client test2 will receive private message ``test message''
\item Client test3 will not see any messages
\end{enumerate}

\subsection*{Clean up}
\begin{enumerate}
\item Hit CTRL+C to kill any remaining clients/servers \dots
\end{enumerate}

% test case 2
\section{Testcase D1002}
\begin{itemize}
\item System: SimpleChat
\item Phase: 3
\item Description: Test creation and use of individual channels.
\end{itemize}

\subsection*{Rationale}
Test to see if clients can create and join channels to send messages through the channel.

\subsection*{Instructions}
\begin{enumerate}
\item Start server
\item Start a client logged in as test
\item Start a client logged in as test2
\item Start a client logged in as test3
\item Run client command from test: ``\#channel newchan''
\item Run client command from test2: ``\#channel newchan''
\item Run client command from test2: ``\#channel newchan test message''
\end{enumerate}

\subsection*{Expected Results}
\begin{enumerate}
\item Client test will create channel \emph{newchan}
\item Client test2 will join existing channel \emph{newchan}
\item Client test will receive channel message ``test message''
\item Client test3 will not see any messages
\end{enumerate}

\subsection*{Clean up}
\begin{enumerate}
\item Hit CTRL+C to kill any remaining clients/servers \dots
\end{enumerate}

% test case 3
\section{Testcase D1003}
\begin{itemize}
\item System: SimpleChat
\item Phase: 3
\item Description: Test client forwarding functionality.
\end{itemize}

\subsection*{Rationale}
Test to see if client can set up forwarding to another client. Forwarding should override any client blocking or channel restrictions by the receiving client.

\subsection*{Instructions}
\begin{enumerate}
\item Start server
\item Start a client logged in as test
\item Start a client logged in as test2
\item Start a client logged in as test3
\item Run client command from test: ``\#channel newchan''
\item Run client command from test2: ``\#channel newchan''
\item Run client command from test3: ``\#block test2''
\item Run client command from test: ``\#forward test2''
\item Run client command from test2: ``\#channel newchan test channel message''
\item Run client command from test2: ``\#private test test private message''
\end{enumerate}

\subsection*{Expected Results}
\begin{enumerate}
\item Client test will create channel \emph{newchan}
\item Client test2 will join existing channel \emph{newchan}
\item Client test3 will block \emph{test2}
\item Client test will start forwarding to \emph{test2}
\item Client test will receive channel message ``test channel message''
\item Client test will receive private message ``test private message''
\item Client test3 will receive forwarded channel message ``test channel message''
\item Client test3 will receive forwarded private message ``test private message''
\end{enumerate}

\subsection*{Clean up}
\begin{enumerate}
\item Hit CTRL+C to kill any remaining clients/servers \dots
\end{enumerate}

% test case 4
\section{Testcase D1004}
\begin{itemize}
\item System: SimpleChat
\item Phase: 3
\item Description: Test client available/unavailable functionality.
\end{itemize}

\subsection*{Rationale}
Test to see if a client can become \emph{\#unavailable} to ignore any messages
and later use \emph{\#available} to get messages again.

\subsection*{Instructions}
\begin{enumerate}
\item Start server
\item Start a client logged in as test
\item Start a client logged in as test2
\item Start a client logged in as test3
\item Run client command from test: ``\#channel newchan''
\item Run client command from test2: ``\#channel newchan''
\item Run client command from test2: ``\#channel newchan test channel message''
\item Run client command from test2: ``\#private newchan test private message''
\item Run client command from test: ``\#unavailable''
\item Run client command from test2: ``\#channel newchan test channel message 2''
\item Run client command from test2: ``\#private test test private message 2''
\item Run client command from test: ``\#available''
\item Run client command from test2: ``\#channel newchan test channel message 3''
\item Run client command from test2: ``\#private test test private message 3''
\end{enumerate}

\subsection*{Expected Results}
\begin{enumerate}
\item Client test will create channel \emph{newchan}
\item Client test2 will join existing channel \emph{newchan}
\item Client test3 will block \emph{test2}
\item Client test will receive channel message ``test channel message''
\item Client test will receive private message ``test private message''
\item Client test will become \emph{unavailable}
\item Client test will \textbf{not} receive channel message ``test channel message 2''
\item Client test will \textbf{not} receive private message ``test private message 2''
\item Client test2 will receive message ``Client test is unavailable''
\item Client test will become \emph{available}
\item Client test will receive channel message ``test channel message 3''
\item Client test will receive private message ``test private message 3''
\item Client test3 will not receive any messages
\end{enumerate}

\subsection*{Clean up}
\begin{enumerate}
\item Hit CTRL+C to kill any remaining clients/servers \dots
\end{enumerate}
\end{document}
